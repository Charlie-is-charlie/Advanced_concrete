\chapter{研究背景}
\label{cha:background}
1.1 传统建筑业的困境与挑战
我国建筑业长期以来依赖以现场浇筑为主的粗放型建造方式。然而,随着城镇化进程的加速推进以及“双碳”目标的提出,这种传统模式正面临着前所未有的严峻挑战。具体来看,首先在资源环境方面,压力巨大。据住房和城乡建设部统计数据显示,建筑业消耗了全国 40%的原材料以及 32%的能源。每平方米建筑会产生约 0.03 立方米的建筑垃圾,全国年产生量超过 20 亿吨。更为严重的是,施工现场扬尘对 PM2.5 的贡献率高达 20% - 30%,已然成为城市空气污染的重要来源之一。其次,从生产效率角度而言,也是不尽如人意。对比分析表明,国内 18 层住宅平均施工周期为 435 天,而采用装配式建造的同类项目仅需 200 天左右。现场作业劳动力密度高达 0.35 人 /㎡,是工业化国家的 3 倍以上。质量合格率长期徘徊在 66%左右,远低于制造业 95%的水平。最后,建筑寿命问题同样不容忽视。我国建筑平均使用寿命不足 30 年,仅为英国的四分之一。过早拆除不仅造成了大量资源的浪费,还产生了大量的建筑垃圾。以上海为例,2022 年建筑垃圾产生量突破 1 亿吨,其中 30%来源于拆除工程。
1.2 装配式混凝土结构的比较优势
预制混凝土结构通过工业化生产方式实现了建造方式的根本变革。在质量提升方面,构件尺寸精度得到极大提高,工厂预制误差能够控制在 ±2mm 以内;混凝土强度离散性显著降低,强度变异系数由 15%降至 5%以下;保护层合格率也从现场施工的 60%提升至 98%。从效率提升来看,以某 30 层住宅项目实测数据为例,标准层施工周期从现浇的 7 天缩短至预制的 3 天,劳动力需求减少了 40%,模板用量降低了 80%。在节能减排效益上,碳排放全生命周期可降低 30% - 40%,建筑垃圾减少 80%以上,施工用水节约 25%。从经济效益角度分析,虽然初期成本会增加约 10% - 15%,但综合考虑工期缩短带来的财务成本节约、质量提升降低的维修费用以及工业化生产带来的人工费下降等因素,全生命周期成本可降低 15% - 20%。
1.3 国内外发展现状比较
1.3.1 国际发展经验
北欧模式以瑞典为例,装配式建筑占比高达 80%,形成了“通用体系”标准化设计。其典型特征包括模数协调,采用 3M 基本模数,构件标准化率超过 90%,并且实现了全专业协同设计。日本体系中,PCa 预制构件协会制定了 JIS 标准,部品通用率在 85%以上。关键技术方面,采用了钢筋套筒灌浆连接、预制抗震墙板体系,同时注重精细化施工管理。德国技术路线则发展了“双皮墙”体系,突出特点为结构与保温一体化、柔性连接节点以及高精度模具系统。
1.3.2 国内发展现状
政策推动方面,2023 年新开工装配式建筑达到 8.24 亿㎡。重点政策包括《“十四五”建筑业发展规划》《装配式建筑评价标准》GB/T51129 以及各地财政补贴政策。技术发展上,我国取得了自主创新成果,如套筒灌浆连接技术、夹心保温外墙体系、模块化建造技术等,但现存瓶颈也不容忽视,节点抗震性能有待提升,自动化生产水平较低,产业协同不足。区域发展存在差异,长三角地区装配率可达 35%,而中西部地区普遍低于 15%。典型城市对比来看,上海是政策强制与市场推动相结合,成都主要依靠政府示范项目引领,深圳则是科技创新驱动。
1.4 发展预制混凝土结构的必要性
从国家战略需求角度,为实现“双碳”目标,建筑业 2030 年前需碳达峰,新建建筑碳排放要降低 40%。新型城镇化建设每年新增建筑 20 亿㎡,产业工人短缺矛盾日益凸显。从行业转型需要来看,我国建筑业劳动生产率较低,仅为 37 万元 / 人年,而美国为 105 万元 / 人年。数字化水平也不高,BIM 技术应用率不足 30%,智能制造装备占比小于 5%。从技术发展趋势而言,建筑工业化与信息化融合、智能建造与新型建筑工业化协同以及绿色低碳技术集成应用是未来发展方向。