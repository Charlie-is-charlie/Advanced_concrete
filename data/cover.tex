\tongjisetup{
  %=========
  % 中文信息
  %=========
  ctitle={基于预制混凝土结构研究进展的\\学习与总结},
  cheadingtitle={基于预制混凝土结构研究进展的学习与总结},   
  cauthor={唐霖辉},  
  studentnumber={2410486},
  ccategories={工学},
  cmajorfirst={土木工程},
  cmajorsecond={结构工程},
  cdepartment={土木工程学院},
  csupervisor={何敏娟 教授}, 
  cresearchfield={大跨木结构},
  cassosupervisor={薛伟辰 教授},
  %=========
  % 英文信息
  %=========
  etitle={Learning and summarizing based on the research progress of precast concrete structures}, 
  eauthor={Linhui Tang},
  ecategories={Gong Xue},
  emajorfirst={Civil Engineering},
  emajorsecond={Structural Engineering},
  edepartment={School of Civil Engineering},    
  esupervisor={Prof. Minjuan He},
  eassosupervisor={Prof. Weichen Xue},
  eresearchfield={Large-span timber construction},
  }
  %=========
  % 中英文摘要和关键字
  %=========
\begin{cabstract}  
预制混凝土结构作为建筑工业化的重要实现形式,具有节能环保、施工高效、质量可控等优势,近年来在建筑与基础设施领域得到广泛应用。本文基于薛伟辰教授的研究成果,系统梳理了预制混凝土结构的研究背景、建筑与基础设施领域的体系研发、规范编制与工程应用,以及当前的研究热点。首先,分析了传统建筑业的挑战及装配式混凝土结构的优势;其次,详细介绍了建筑领域的框架结构、剪力墙结构、全装配式框架-剪力墙结构及夹心保温外墙的研究进展;随后,探讨了基础设施领域的综合管廊、核电厂结构及地铁车站等预制混凝土结构的应用;此外,总结了国内外相关规范的编制情况及典型工程案例;最后,展望了未来研究方向,包括模块化建筑、智能建造、高性能材料应用等。本文旨在为预制混凝土结构的进一步研究与应用提供参考。
\end{cabstract}

\ckeywords{预制混凝土结构;装配式建筑;抗震性能;智能建造}

\begin{eabstract}
As an important realization form of building industrialization, precast concrete structure, with the advantages of energy saving, environmental protection, efficient construction and controllable quality, has been widely used in the field of construction and infrastructure in recent years. Based on the research results of Prof. Xue Weichen, this paper systematically combs through the research background of precast concrete structures, the system development, specification preparation and engineering application in the field of construction and infrastructure, as well as the current research hotspots. Firstly, it analyzes the challenges of the traditional construction industry and the advantages of assembled concrete structures; secondly, it introduces in detail the research progress of frame structures, shear-wall structures, fully assembled frame-shear-wall structures, and sandwich insulated facades in the field of construction; subsequently, it discusses the application of precast concrete structures in the field of infrastructure, such as comprehensive pipeline corridors, structures of nuclear power plants, and subway stations; in addition, it summarizes the relevant domestic and international normative preparation and typical engineering cases; finally, it looks forward to future research directions, including modular building, intelligent construction, and application of high-performance materials. This paper aims to provide reference for further research and application of precast concrete structures.
\end{eabstract}

\ekeywords{precast concrete structures; assembled buildings; seismic performance; intelligent construction}